\documentclass[12pt]{article}

%% preamble: Keep it clean; only include those you need
\usepackage{amsmath}
\usepackage[margin = 1in]{geometry}
\usepackage{graphicx}
\usepackage{booktabs}
\usepackage{natbib}
\usepackage[T1]{fontenc}


% for space filling
\usepackage{lipsum}
% highlighting hyper links
\usepackage[colorlinks=true, citecolor=blue]{hyperref}


%% meta data

\title{Analyzing Player Performance in Premier League Soccer Using 
Data Visualization and Predictive Analysis}
\author{Okem Chime\\
  University of Connecticut
}

\begin{document}
\maketitle

\begin{abstract}
This is the abstract. The purpose of this part of the research paper is to
summarise the paper.

\end{abstract}


\section{Introduction}
\label{sec:intro}


%The introduction is used to a refernce framework for the reader. 
%The aim of this section is to answer three questions:
%Why is the topic important/interesting?
%What has been done on this topic in the literature?
%What is your contribution?


For many people in the world, sports are more than simple games or recreational
activities, forming a significant part of our lives and culture. For hundreds of
years sports have acted as a tool to transcend the boundaries of geography,
language and culture. Global sporting events such as the Olympic games and the
World Cup assert the significance of sport in unifying people worldwide.
Finacially, the sports industry is a powerhouse, with signifcant economic and
social impacts. It drives revenue, tourism and job oppportunities, making it a 
critical componenet of many nation's economies. 

Performance data is a large and constantly growing field of data science. With
so many aspects of professional sport that can be analyzed, sport is an 
extremely relevant domain where data science techniques can be applied in order
to give players and teams a potential edge. Soccer is the most popular sport in
the world with more than 3.5 billion fans around the globe and over 250 million
players in over 200 countries.

Amongst this, the English Premier League is the most watched sports league in
the world with a potential viewership of over 4.7 billion people. 
With so much attention focused specfically on this league,
players, coachhes, fans sponsors and more are dependent on the success of the
league as a whole as well as how specific teams and players are performing.

This is where data science in the form of sports performance analysis comes in.
This domain has revolutionized the world of performance by adding a new layer to
the sporting world, offering insights which were previously unimaginable. 
These  statistical insights can guide decisions that can make all the difference 
at all levels of the game; all the way from the players on the pitch to the fans
in the stadium.

There has been recent work done by \citet{goes2021unlocking} which takes into 
account the importance of big data data in the professional soccer setting. 
With the increase in technology available to professional soccer players we have
seen the use of wearable position tracking devices which have been widely 
adopted by professional sports organizations with the understanding that they
could be a game changer in the world of high peformance sport. The focus of this 
work focuses more on the tactical implications of this type of data whereas
our research will be more focused on how performance data can be used to guide
decisions about players.

Some interesting work was also done that used performance data as a metric to 
determine player value. This study looked at s variety of factors in order to 
determine player value. Some of these metrics included on-field statistics and
other fairly obvious factors such as discipliniary record and injuries as well
less obvious variables such as image rights and personal background etc 
\citep{tunaru2010valuations}. This study like ours, uses OPTA data to look
evaluate performance at a player level in order to help formulate decisions on
players. However, this study incorporates off-field metrics as well as  objective
peformance data to inform their financial evulations on players whereas our
study will be only based on objective on field performance data and will attempt
to provide more insights into their current level of performance as well as
their predicted future level of performance.


The objective of this paper will be to investigate how data visualization and
predictive analysis can be used to assess and improve player performance in
Premier League soccer. It then seeks to answer the quesion:
"How does historical player performance data, including key performance
metrics influence the prediction of player's goal contributions
(combinecd goals and assists) per 90 minutes in future Premier League 
soccer matches, and how can this information be leveraged for strategic 
decision-making?". 
By analysisng the performance statistics of premier league 
players from previous years, this paper will able to decipher the important
factors that are significant in contributing to a player's total goal
contributions per 90 minutes in a premier league season. From this, we will use 
statisical techniques to build a model which can predict a player's goal 
contributions per 90 minutes for a specific premier league season on the basis
of their previous performance history. 
This study will focus on a subset of 9 
midfield and attacking players who have competed in all 10 of the most recent 
premier league seasons. Goal contributions are an important metric within soccer
because of the simple fact that goals are the single factor which determines
which team will win a soccer match. 



% roadmap
The rest of the paper is organized as follows.
The data will be presented in Section~\ref{sec:data}.
The methods are described in Section~\ref{sec:meth}.
The results are reported in Section~\ref{sec:resu}.
A discussion concludes in Section~\ref{sec:disc}.


\section{Data}
\label{sec:data}


% Utilize this segment to provide an account of the data essential for addressing
% your research inquiries. 



\section{Methods}
\label{sec:meth}


% Utilize this section to introduce the methodologies that will produce results 
% through the analysis of the provided data.





\section{Results}
\label{sec:resu}


\section{Discussion}
\label{sec:disc}



% What are the main contributions again?

% What are the limitations of this study?

% What are worth pursuing further in the future?






\bibliography{refs}
\bibliographystyle{plainnat} 


\end{document}