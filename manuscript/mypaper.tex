\documentclass[12pt]{article}

%% preamble: Keep it clean; only include those you need
\usepackage{amsmath}
\usepackage[margin = 1in]{geometry}
\usepackage{graphicx}
\usepackage{booktabs}
\usepackage{natbib}
\usepackage[T1]{fontenc}


% for space filling
\usepackage{lipsum}
% highlighting hyper links
\usepackage[colorlinks=true, citecolor=blue]{hyperref}


%% meta data

\title{Analyzing Player Performance in Premier League Soccer Using 
Data Visualization and Predictive Analysis}
\author{Okem Chime\\
  University of Connecticut
}

\begin{document}
\maketitle

\begin{abstract}
This is the abstract. The purpose of this part of the research paper is to
summarise the paper.

\end{abstract}


\section{Introduction}
\label{sec:intro}


%The introduction is used to a refernce framework for the reader. 
%The aim of this section is to answer three questions:
%Why is the topic important/interesting?
%What has been done on this topic in the literature?
%What is your contribution?


For many people worldwide, sports transcend the realms of mere games or leisure
activities, acting as a unifying force that transcends geographical, linguistic,
and cultural boundaries. Global sporting events, exemplified by the Olympic Games
and the World Cup, underscore the profound significance of sports in fostering
global unity. Beyond its cultural impact, the sports industry wields substantial
financial power, exerting significant economic and social influences. It serves
as a catalyst for revenue generation, tourism influx, and job opportunities,
playing a pivotal role in the economies of many nations.

Within the vast landscape of data science, sports performance data represents a
burgeoning and expansive field. The intricate details of professional sports
provide a fertile ground for data science techniques, offering a potential
competitive edge to players and teams. Soccer, being the world's most popular
sport with over 3.5 billion fans globally and more than 250 million players
across 200 countries, stands as a prime arena for data-driven insights.

At the epicenter of this global passion for soccer is the English Premier League,
a sports spectacle watched by millions worldwide, boasting a potential viewership
exceeding 4.7 billion individuals. The league's immense popularity makes it a
critical focal point for players, coaches, fans, sponsors, and stakeholders alike.
The success of both the league as a whole and the individual performances of teams
and players significantly impacts various facets of the sport.

In this high-stakes environment, data science emerges as a game-changer through
sports performance analysis. This domain revolutionizes the sporting landscape
by introducing a new layer of insights that were once unimaginable. The
statistical revelations derived from data science have the potential to guide
decisions at all levels of the game – from players on the pitch to the fans in
the stadium. By delving into the intricacies of player performance, tactics, and
team dynamics, sports performance analysis transforms raw data into actionable
intelligence, offering a nuanced understanding of the game that goes beyond
traditional observations.


There has been recent work done by \citet{goes2021unlocking} which takes into 
account the importance of big data data in the professional soccer setting. 
With the increase in technology available to professional soccer players we have
seen the use of wearable position tracking devices which have been widely 
adopted by professional sports organizations with the understanding that they
could be a game changer in the world of high peformance sport. The focus of this 
work focuses more on the tactical implications of this type of data whereas
our research will be more focused on how performance data can be used to guide
decisions about players.

Some interesting work was also done that used performance data as a metric to 
determine player value. This study looked at s variety of factors in order to 
determine player value. Some of these metrics included on-field statistics and
other fairly obvious factors such as discipliniary record and injuries as well
less obvious variables such as image rights and personal background etc 
\citep{tunaru2010valuations}. This study like ours, uses OPTA data to look
evaluate performance at a player level in order to help formulate decisions on
players. However, this study incorporates off-field metrics as well as  objective
peformance data to inform their financial evulations on players whereas our
study will be only based on objective on field performance data and will attempt
to provide more insights into their current level of performance as well as
their predicted future level of performance.


The objective of this paper will be to investigate how data visualization and
predictive analysis can be used to assess and improve player performance in
Premier League soccer. It then seeks to answer the quesion:
"How does historical player performance data, including key performance
metrics influence the prediction of player's goal contributions
(combinecd goals and assists) per 90 minutes in future Premier League 
soccer matches, and how can this information be leveraged for strategic 
decision-making?".

By analysisng the performance statistics of premier league 
players from previous years, this paper will able to decipher the important
factors that are significant in contributing to a player's total goal
contributions per 90 minutes in a premier league season. From this, we will use 
statistical techniques to build a model which can predict a player's goal 
contributions per 90 minutes for a specific premier league season on the basis
of their previous performance history. 

This study will focus on a subset of 9 
midfield and attacking players who have competed in all 10 of the most recent 
premier league seasons. Goal contributions are an important metric within soccer
because of the simple fact that goals are the single factor which determines
which team will win a soccer match. Assists are also highly valued as they are
the vital previous step which leads to a goal and are thus an important metric
in attacking play.

Due to the fact that attacking players
play further up the field they are expected to primarily contribute goals to the 
team, as well as some assists.
Midfielders are expected to cover all areas of the field, thus due to having
more defensive responsibilites than attackers, while still expected to
contribute some amount of goals, due to their role in connected defence to
attack are expected to primarily provide assists. 

Thus, the metric of combined goals and assists (known as 'goal contributions') is 
a very good statisitc to measure a player's overall attacking productivity.
By including a player's assists it somewhat removes the bias of attacking players
to achieve better attacking productivity due to their advantage of them playing 
much higher up the field- although not completely. Goals contributions per 90 
minutes is the performance metric primarily focused on this paper because it
aids in removing the bias achieved from a player having played a greater amount 
of minutes than another, due to the fadct that the more minutes you play, the
more opportunity you have to score or assist a goal.



% roadmap
The rest of the paper is organized as follows.
The data will be presented in Section~\ref{sec:data}.
The methods are described in Section~\ref{sec:meth}.
The results are reported in Section~\ref{sec:resu}.
A discussion concludes in Section~\ref{sec:disc}.


\section{Data}
\label{sec:data}


% Utilize this segment to provide an account of the data essential for addressing
% your research inquiries. 

In this study, the primary source of data is the Premier League OPTA performance
data obtained from the reputable and comprehensive website, Fbref.com.
Fbref.com provides detailed player and team statistics for various soccer
leagues, including the Premier League, making it a reliable foundation for our
research.

The dataset encompasses a wide range of metrics that collectively form a
comprehensive overview of a player's on-field performance. These metrics include
fundamental statistics such as matches played, goals scored, and assists.
Additionally, the dataset incorporates more intricate metrics like expected goals
and assists, which offer valuable insights into the nuanced aspects of player
performance.

To ensure a robust and extensive analysis, we will be utilizing data spanning
ten seasons, from the 2013-14 Premier League season to the 2022-23 season.
This decade-long timeframe enables us to capture a substantial sample of player
performances, allowing for a thorough examination of trends and patterns over
the years. Using 10 seasons of performaance data also provides a quantity of 
data extensive enough to draw reliable insights from as well as provide enough
data input to use for our models for the predictive portion of the research.
On average each competitive season provides approximately 530 observations-
representing the number of players that competeted in that Premier League 
season. For each observation, the data from seasons 2013-14 until 2016-17 contain
23 variables whilst from seasons 2017-18 until 2022-23 contains 35 variables.
The additional variables represented in the more recent Premier League seasons
represents the performance metrics associated with 'expected goals', which was
not a commonly adopted performance statistic in earlier Premier League seasons.





\section{Methods}
\label{sec:meth}


% Utilize this section to introduce the methodologies that will produce results 
% through the analysis of the provided data.





\section{Results}
\label{sec:resu}


\section{Discussion}
\label{sec:disc}



% What are the main contributions again?

% What are the limitations of this study?

% What are worth pursuing further in the future?






\bibliography{refs}
\bibliographystyle{plainnat} 


\end{document}