\documentclass[12pt]{article}

%% preamble: Keep it clean; only include those you need
\usepackage{amsmath}
\usepackage[margin = 1in]{geometry}
\usepackage{graphicx}
\usepackage{booktabs}
\usepackage{natbib}
\usepackage[T1]{fontenc}

% highlighting hyper links
\usepackage[colorlinks=true, citecolor=blue]{hyperref}

\title{Proposal: Analyzing Player Performance in Premier League Soccer Using 
Data Visualization and Predictive Analysis }
\author{Okem Chime\\
  University of Connecticut
}

\begin{document}
\maketitle

\paragraph{Introduction}
Performance data is a large and constantly growing field of data science. With
so many aspects of professional sport that can be analyzed, sport is an 
extremely relevant domain where data science techniques can be applied in order
to give players and teams a potential edge. Soccer is the most popular sport in
the world with more than 3.5 billion fans around the globe and over 250 million
players in over 200 countries. Amongst this, the English Premier League is the 
most watched sports league in the world with a potential viewership of over 4.7 
billion people. With so much attention focused specfically on this league,
players, coachhes, fans sponsors and more are dependent on the success of the
league as a whole as well as how specific teams and players are performing. 
This is where data science comes in as a tool to amalgmate information,
provide statistical insights and guide decisions that can make all the difference 
at all levels of the game; all the way from the players on the pitch to the fans
in the stadium.

There has been recent work done by \citet{goes2021unlocking} which takes into 
account the importance of big data data in the professional soccer setting. 
With the increase in technology available to professional soccer players we have
seen the use of wearable position tracking devices which have been widely 
adopted by professional sports organizations with the understanding that they
could be a game changer in the world of high peformance sport. The focus of this 
work focuses more on the tactical implications of this type of data whereas
our research will be more focused on how performance data can be used to guide
decisions about players.

Some interesting work was also done that used performance data as a metric to 
determine player value. This study looked at s variety of factors in order to 
determine player value. Some of these metrics included on-field statistics and
other fairly obvious factors such as discipliniary record and injuries as well
less obvious variables such as image rights and personal background etc 
\citep{tunaru2010valuations}. This study like ours, uses OPTA data to look
evaluate performance at a player level in order to help formulate decisions on
players. However, this study incorporates off-field metrics as well as  objective
peformance data to inform their financial evulations on players whereas our
study will be only based on objective on field performance data and will attempt
to provide more insights into their current level of performance as well as
their predicted future level of performance.



\paragraph{Specific Aims}

The research question that we will be attempting to answer through this paper is
"How can data visualization and predictive analysis be used to assess and 
improve player performance in the Premier League soccer?"

More specficially. statistical/data science question we will be aiming to answer
is: "How does historical player performance data, including key performance
metrics influence the prediction of player performance in future Premier League 
soccer matches, and how can this information be leveraged for strategic 
decision-making?"

This question focuses on the predictive aspect of my research and combines
elements of data analysis, statistical modeling, and predictive analytics. 
It also underscores the practical application of my analysis in strategic 
decision-making within the Premier League context.




\paragraph{Data}
The primary source of data for this research is Premier league OPTA performance 
data obtained from Fbref.com, which is a reputable and comprehensive website 
that provides detailed player and team statistics for various soccer leagues, 
including the Premier League.

These metrics cover a player's on-field performance and include metrics such as
matches played, goals scored, assists etc. 

I will be using 10 seasons of data, ranging from the 2013-14 Premier League
season up to the 2022-23 premier league season 




\paragraph{Research Design and Methods}
What design or methods will you use?
Cite relevant references~ %\citep[e.g.,][]{wild2004global}.



\paragraph{Discussion}
What are the most challenge parts of the task?
What are the limitations of your work? What is your fall-back plan if
something unexpected happens?



\bibliography{refs1}
\bibliographystyle{chicago}

\end{document}



\end{document}