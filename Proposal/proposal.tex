\documentclass[12pt]{article}

%% preamble: Keep it clean; only include those you need
\usepackage{amsmath}
\usepackage[margin = 1in]{geometry}
\usepackage{graphicx}
\usepackage{booktabs}
\usepackage{natbib}
\usepackage[T1]{fontenc}

% highlighting hyper links
\usepackage[colorlinks=true, citecolor=blue]{hyperref}

\title{Proposal: Analyzing Player Performance in Premier League Soccer Using 
Data Visualization and Predictive Analysis }
\author{Okem Chime\\
  University of Connecticut
}

\begin{document}
\maketitle

\paragraph{Introduction}
Performance data is a large and constantly growing field of data science. With
so many aspects of professional sport that can be analyzed, sport is an 
extremely relevant domain where data science techniques can be applied in order
to give players and teams a potential edge. Soccer is the most popular sport in
the world with more than 3.5 million fans around the globe and over 250 million
players in over 200 countries. Amongst this, the English Premier League is the 
most watched sports league in the world with a potential viewership of over 4.7 
billion people. With so much attention focused specfically on this league,
players, coachhes, fans sponsors and more are dependent on the success of the
league as a whole as well as how specific teams and players are performing. 
This is where data science comes in as a tool to amalgmate information,
provide statistical insights and guide decisions that can make all the difference 
at all levels of the game; all the way from the players on the pitch to the fans
in the stadium.


\paragraph{Specific Aims}
Formulate your research question;
translate your research question into statistical/data science questions


\paragraph{Data}
Hopefully, you have identified the data needed for your project. Give a
description about it.



\paragraph{Research Design and Methods}
What design or methods will you use?
Cite relevant references~\citep[e.g.,][]{wild2004global}.



\paragraph{Discussion}
What are the most challenge parts of the task?
What are the limitations of your work? What is your fall-back plan if
something unexpected happens?



\bibliography{../manuscript/refs}
\bibliographystyle{chicago}

\end{document}



\end{document}